\documentclass{article}
\usepackage{graphicx} % Required for inserting images
\usepackage[top=0.9in, bottom=1in, left=1.5in, right=1.5in]{geometry}
\usepackage[utf8]{inputenc}
\usepackage[icelandic]{babel}
\usepackage[T1]{fontenc}
\usepackage[sc]{mathpazo}
\usepackage[parfill]{parskip}
\renewcommand{\baselinestretch}{1.2}
\usepackage{booktabs,tabularx}
\usepackage{multirow}
\usepackage{enumerate}
\usepackage{adjustbox}
\usepackage{multicol}
\usepackage{xcolor}
\usepackage{algpseudocode}
\usepackage{tikz}
\usepackage{nicefrac}
\usepackage{changepage}
\usetikzlibrary{arrows, positioning, calc, graphs}
\usepackage{amsmath, amsfonts, amssymb, amsthm}
\usepackage{graphicx}
\usepackage{tikz}
\usepackage{minted}
\usemintedstyle{manni}
\title{Gamer}
\author{Ragnar Björn Ingvarsson, rbi3}
\tikzset{->, >=stealth', shorten >=1pt, node distance=2cm,thick, main node/.style={circle,draw,minimum size=3em}}

\begin{document}
\renewcommand\thepage{}

	\maketitle

	\newpage
	\setcounter{page}{1}
	\renewcommand\thepage{\arabic{page}}

	\section{}
	\section{Reiknirit fer í tvöfalda lykkju, þá ytri með k frá 1 til n og þá innri með i frá 1 til k. Inni í lykkjunni eru framkvæmdar i aðgerðir. Hve margar aðgerðir framkvæmir reikniritið alls?}
	Við setjum upp í summur:
	\[
		\sum_{k=1}^{n}\sum_{i=1}^k i
	\]
	Og notum summuformúluna
	\begin{equation}
		\sum_{k=1}^n k = \frac{n(n+1)}{2}
		\label{eq:gamer1}
	\end{equation}
	Og þá reiknum við upp úr innri summunni fyrst og fáum
	\[
		\sum_{k=1}^{n}\sum_{i=1}^k i = \sum_{k=1}^n \frac{k(k+1)}{2}
		= \frac{1}{2}\left(\sum_{k=1}^n k^2 + \sum_{k=1}^n k\right)
	\]
	Munum eftir summuformúlu fyrir $k^2$
	\begin{equation}
		\sum_{k=1}^n k^2 = \frac{n(n+1)(2n+1)}{6}
		\label{eq:gamer2}
	\end{equation}
	Og svo notum við bæði (1) og (2) og fáum loks
	\[
		= \frac{1}{2}\left( \frac{n(n+1)(2n+1)}{6} + \frac{n(n+1)}{2}\right)
		= \frac{n(n+1)(2n+1) + 3n(n+1)}{12}
	\]
	\[
		= \frac{n(n+1)(2n+4)}{12} = \frac{n(n+1)(n+2)}{6} 
	\]

	\vspace{-1.5em}$\hfill\square$

\end{document}

\documentclass{article}
\usepackage{graphicx} % Required for inserting images
\usepackage[top=0.9in, bottom=1in, left=1.5in, right=1.5in]{geometry}
\usepackage[utf8]{inputenc}
\usepackage[icelandic]{babel}
\usepackage[T1]{fontenc}
\usepackage[sc]{mathpazo}
\usepackage[parfill]{parskip}
\renewcommand{\baselinestretch}{1.2}
\usepackage{booktabs,tabularx}
\usepackage{multirow}
\usepackage{enumerate}
\usepackage{adjustbox}
\usepackage{multicol}
\usepackage{xcolor}
\usepackage{algpseudocode}
\usepackage{tikz}
\usepackage{nicefrac}
\usepackage{changepage}
\usetikzlibrary{arrows, positioning, calc, graphs}
\usepackage{amsmath, amsfonts, amssymb, amsthm}
\usepackage{graphicx}
\usepackage{tikz}
\usepackage{minted}
\usemintedstyle{manni}
\title{Tölvutækni og Forritun Heimadæmi 10}
\author{Ragnar Björn Ingvarsson, rbi3}
\tikzset{->, >=stealth', shorten >=1pt, node distance=2cm,thick, main node/.style={circle,draw,minimum size=3em}}

\begin{document}
\renewcommand\thepage{}

	\maketitle

	\newpage
	\setcounter{page}{1}
	\renewcommand\thepage{\arabic{page}}

	\section{}
	\begin{itemize}
		\item[a)] Við byrjum með \texttt{0x2051} og framkvæmum svo 
			\texttt{0x2051 | 12} (or-að við 12) sem gefur, ef við breytum yfir í 
			tvíundakerfið, \texttt{0010000001010001 | 1100} svo við fáum 
			\texttt{0010000001011101} sem, ef við svo framkvæmum \texttt{>{}> 3} (hliðrum um 3 til vinstri) 
			á það fæst \texttt{10000001011} eða \texttt{0x40b}.
		\item[b)] 
			\begin{itemize}
				\item[i.] Við sjáum að $13 = 1101_2$ og þá rennum við einu 
					sinni í gegn og þá er \texttt{v = 1100}, rennum svo 
					aftur og \texttt{v = 1000} og svo loks rennum við 
					einu sinni enn og fæst \texttt{v = 0000} og við 
					hættum. Svo í lokin er \texttt{v = 0} og 
					\texttt{c = 3}.
				\item[ii.] Almennt sjáum við að \texttt{v = 0} í lokin 
					þar sem lykkjan hættir bara þegar \texttt{v} gefur 
					falsgildið $0$. Svo verður \texttt{c} bara fjöldi ása 
					í tvíundartölunni fyrir \texttt{v}.
			\end{itemize}
	\end{itemize}

	\section{}
	Sjáum að \texttt{i} er í \texttt{rdi}, \texttt{j} er í \texttt{rsi} og 
	\texttt{k} er í \texttt{rdx}. Fyrst er \texttt{k} fært í skilagildi og 
	við byrjum á samanburði á \texttt{i} og \texttt{j} og framkvæmum línu 
	\texttt{8} ef sannur, annars hoppum við beint í annan samanburð sem 
	aftur framkvæmir bara eina línu, \texttt{10}, ef sannur. Svo kóðinn er 
	svona
	\begin{minted}{c}
int bla(int i, int j, int k) {
	if (i >= j) {
	    j = i;
	}
	if (j >= k) {
	    return j;
	}
	return k;
}
	\end{minted}

	\section{}
	\begin{itemize}
		\item[a)] Byrjum á \texttt{main.c}, \texttt{fun} er veikt, 
			\texttt{b} er veikt vegna \texttt{extern}, \texttt{c} er sterkt 
			og \texttt{main} er sterkt.

			Í \texttt{fun.c} er \texttt{a} sterkt, \texttt{b} líka sterkt, 
			\texttt{c} veikt vegna \texttt{extern} og \texttt{fun} er sterkt.
		\item[b)] Það prentast út að a sé 0, b sé 3 og c sé 5. 
			Þetta er líklega vegna þess að þegar við setjum \texttt{b = 3} 
			þá er það að virka sem \texttt{int} í stað \texttt{short int} 
			svo breytan í \texttt{fun.c} fær neðri partinn, en efri 
			breytir óvart \texttt{short int a} í staðinn og \texttt{a} 
			verður þá $0$.
	\end{itemize}

	\section{}
	\begin{itemize}
		\item[a)] Kyrrleg tenging getur verið betri þar sem óþarfi er að 
			hafa öll söfnin sem forrit notar á tölvunni, þ.e. allt sem 
			þarf fyrir forritið kemur með keyrsluskránni. Einnig veldur 
			þetta því að kyrrleg tenging er fljótari þar sem ekki þarf að 
			tengja söfn þegar forritinu er hlaðið í minni eða í miðri 
			keyrslu, allt er nú þegar komið þegar forritið er þýtt.
		\item[b)] Kvik tenging er oft betri vegna þess t.d. að 
			keyrsluskrárnar verða miklu minni vegna þess að ekki þarf að 
			pakka öllum notkunum í söfnum saman við forritið. Einnig er 
			auðveldara að uppfæra söfn þar sem ekki þarf að endurþýða öll 
			forrit sem nota það í hvert skipti sem það er uppfært. Svo 
			geta líka mörg forrit notað sama minni fyrir safnið í einu.
	\end{itemize}

	\section{}
	\begin{itemize}
		\item[a)] Við sjáum að hér kemur villa þar sem eftir kallið á 
			\texttt{f()} þá breytist \texttt{x} ekki í \texttt{a1.c} og 
			helst sem $2$, í stað þess að breytast í $4$. Þetta er vegna 
			þess að þó við skilgreinum \texttt{extern int x} sem tengist 
			við \texttt{x}-ið í \texttt{a2.c} þá er þessi skilgreining veik 
			og þess vegna notum við frekar \texttt{x}-ið skilgreint í 
			\texttt{int x = 2} þar sem það er sterk skilgreining sem tekur 
			forgang yfir hina skilgreininguna.
		\item[b)] Hér er engin villa þar sem eina skilgreiningin á \texttt{y} 
			í \texttt{b1.c} er \texttt{int y = 10} sem er sterk og við 
			vísum í hana frá \texttt{b2.c} með \texttt{extern int y}. 
			Þess vegna er engin tvískilgreining og allt er hreint og skýrt 
			og virkar eins og við búumst við, \texttt{y} byrjar sem $10$ og 
			prentast út, verður svo $12$ og prentast aftur út.
	\end{itemize}
\end{document}

\documentclass{article}
\usepackage{graphicx} % Required for inserting images
\usepackage[top=0.9in, bottom=1in, left=1.5in, right=1.5in]{geometry}
\usepackage[utf8]{inputenc}
\usepackage[icelandic]{babel}
\usepackage[T1]{fontenc}
\usepackage[sc]{mathpazo}
\usepackage[parfill]{parskip}
\renewcommand{\baselinestretch}{1.2}
% Tables and lists
\usepackage{booktabs,tabularx}
\usepackage{multirow}
\usepackage{enumerate}
\usepackage{adjustbox}
\usepackage{multicol}
\usepackage{xcolor}
\usepackage{algpseudocode}
\usepackage{algorithm}
\usepackage{tikz}
\usepackage{nicefrac}
\usepackage{changepage}
\usepackage{fancyvrb}
\usepackage{xlop}
\usetikzlibrary{arrows, positioning, calc, graphs}

% Math
\usepackage{amsmath, amsfonts, amssymb, amsthm}
% Graphics

\usepackage{graphicx}
\usepackage{tikz}
% Code environment
\usepackage{minted}
%\usepackage{bm}
%\usepackage{siunitx}
%\usepackage{animate}
%\usepackage{hyperref}
%\usepackage{movie15}
%\usepackage{multicol}
%\usepackage{changepage}
\title{Tölvutækni og forritun Heimadæmi 6}
\author{Ragnar Björn Ingvarsson, rbi3}
\tikzset{->, >=stealth', shorten >=1pt, node distance=2cm,thick, main node/.style={circle,draw,minimum size=3em}}


\begin{document}
\renewcommand\thepage{}
	
	\maketitle

	\newpage
	\setcounter{page}{1}
	\renewcommand\thepage{\arabic{page}}

	\section{Segjum að gistið $\%rax$ innihaldi $0x200$ og gistið $\%rcx$ 
		innihaldi $0x5$. Í hverju tilviki reiknið út raunverulegt vistfang.}
	\begin{itemize}
		\item[a)] $\$0x18(\%rax)$ Fáum hér $0x18$ sem hliðrun fyrir gildi 
			gistisins $\%rax$ sem er þá $0x18 + 0x200 = 0x218$
		\item[b)] $(\%rax,\%rcx,8)$ Hér plúsum við saman gildi $\%rax$ og 
			áttfalt gildi $\%rcx$, svo $0x200 + 8 * 0x5 = 0x200 + 0x28 = 
			0x228$
		\item[c)] $\$50(,\%rax,4)$ Hér er hliðrun um $50$ á fjórfalt gildi 
			$\%rax$, svo $50 + 4 * 0x200 = 0x32 + 0x800 = 0x832$
	\end{itemize}

	\section{Hér eru gagnaflutningsskipanir fyrir mismunandi gagnastærðir. 
		Það vantar hins vegar síðasta bókstafinn í skipanirnar eftir því 
		hvaða gagnastærðir er verið að flytja. Skrifið skipanirnar aftur 
		upp með réttum síðasta bókstaf í stað spurningamerkisins.}

		\begin{itemize}
			\item[] $movl \hspace{1em} (\%rbx, \%rax, 2), \%edx$
			\item[] $movq \hspace{1em} \$2, \%rdi$
			\item[] $movb \hspace{1em} \%ah, (\%rcx)$
			\item[] $movw \hspace{1em} \%bx, \%dx$
		\end{itemize}

	\section{Hér fyrir neðan eru nokkrar einfaldar segðir. Sýnið ódýrar 
	smalamálsskipanir til að reikna þær. Gerið ráð fyrir að breytan $x$ sé 
	í gistinu $\%rax$, en þið megið nota önnur gisti ef þið þurfið.}

	\begin{itemize}
		\item[a)] $leaq \hspace{1em} (\%rax, \%rax, 4), \%rax$
		\item[b)] $leaq \hspace{1em} (\%rax, \%rax, 8), \%rax$ \\
				  $leaq \hspace{1em} (\%rax, \%rax, 2), \%rax$
		\item[c)] $leaq \hspace{1em} (\%rax, \%rax, 8), \%rax$ \\
				  $leaq \hspace{1em} (\%rax, \%rax, 4), \%rax$
		\item[d)] $leaq \hspace{1em} (\%rax, \%rax), \%rdx$ \\
				  $leaq \hspace{1em} (\%rax, \%rdx, 5), \%rax$
	\end{itemize}

	\section{Hér fyrir neðan er smalamálsútgáfa af fallinu \\ $long 
		\hspace{0.3em} reikn(long \hspace{0.3em}x, \hspace{0.3em}long 
	 \hspace{0.3em}y)$ sem reiknar einfalda segð frá viðföngunum $x$ og $y$. 
 Athugið að viðfangið $x$ kemur í gistið $\%rdi$, viðfangið $y$ kemur í 
$\%rsi$ og skilagildi fallsins fer í gistið $\%rax$.}

	\begin{itemize}
		\item[a)] Hér er útskýring \vspace{1em} \\ $\hspace{2em} reikn:$ \\
			$\text{Hér erum við að margfalda y með 5 og setja það í gistið 
			\%rax}$ \\
	$- \hspace{2em} leaq \hspace{2em} (\%rsi,\%rsi,4),\hspace{1em}\%rax$ \\
			$\text{Hér tökum við 5y og bætum við það x margfaldað með 
			tveimur, og setjum þá 5y + 2x í \%rdx}$ \\
	$- \hspace{2em} leaq \hspace{2em} (\%rax,\%rdi,2),\hspace{1em}\%rdx$ \\
			$\text{Tökum svo 5y + 2x og margföldum það með 8 og setjum í 
			skilagildið \%rax}$ \\
	$- \hspace{2em} leaq \hspace{2em} 0(,\%rdx,8),\hspace{1em}\%rax$ \\
			$\text{Drögum svo 5y + 2x einu sinni frá skilagildinu til að 
			fá loks 7(5y + 2x)}$ \\
	$- \hspace{2em} subq \hspace{2em} \%rdx,\hspace{1em}\%rax$ \\
			$\text{Skilum svo 7(5y + 2x)}$ \\
	$- \hspace{2em} ret$
		\item[b)] 
			\begin{verbatim}
long reikn(long x, long y) {
    return 7*(5*x + 2*y);
}
			\end{verbatim}
	\end{itemize}

	\section{Þið fáið gefna Linux keyrsluskránna $haha$. Hún hefur 
		$main$-fall sem tekur inn eina heiltölu á skipanalínunni, kallar á 
		fallið $hvad$ og prentar út skilagildi þess. Í þessu dæmi eigið þið 
		að endurskrifa fallið í C kóða og útskýra í orðum hvað það gerir.}

		Við sjáum að assembly kóðinn fyrir fallið lítur svona út:
		\begin{verbatim}
0000000000001169 <hvad>:
    1169:	f3 0f 1e fa          	endbr64
    116d:	8d 04 3f             	lea    (%rdi,%rdi,1),%eax
    1170:	21 f8                	and    %edi,%eax
    1172:	c1 e7 02             	shl    $0x2,%edi
    1175:	21 f8                	and    %edi,%eax
    1177:	c3                   	ret
		\end{verbatim}
		
		Þar sem inntaksgildi fallsins n er í \%edi er \%rdi bara 64-bita 
		útgáfa af því svo við fyrst tvöföldum n og setjum í \%eax. Síðan 
		bitwise AND-um við þá n og 2n og geymum niðurstöðuna í \%eax.

		Svo bit shiftum við n um 2 til vinstri og geymum í \%edi, og loks 
		bitwise AND-um við \%edi, semsagt n<<2, við \%eax, semsagt n \& 2n.

		Svo skilum við því gildi, þ.e. \%eax.

		Þá getum við skrifað C kóðann

		\begin{verbatim}
int hvad(unsigned int n) {
    return n & 2*n & n<<2;
}
		\end{verbatim}

		Þar sem við fyrst AND-um saman n og 2n, og svo niðurstöðuna úr því 
		við n shiftað um 2 til vinstri. Fallið virkar þá þannig að það fær 
		inn tölu og skilar bitastreng sem inniheldur ás allstaðar þar sem 
		þrír ásar koma fram í inntakinu í röð til hægri.


\end{document}

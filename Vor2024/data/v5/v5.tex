\documentclass{article}
\usepackage{graphicx} % Required for inserting images
\usepackage[top=0.9in, bottom=1in, left=1.5in, right=1.5in]{geometry}
\usepackage[utf8]{inputenc}
\usepackage[icelandic]{babel}
\usepackage[T1]{fontenc}
\usepackage[sc]{mathpazo}
\usepackage[parfill]{parskip}
\renewcommand{\baselinestretch}{1.2}
% Tables and lists
\usepackage{booktabs,tabularx}
\usepackage{multirow}
\usepackage{enumerate}
\usepackage{adjustbox}
\usepackage{multicol}
\usepackage{xcolor}
\usepackage{algpseudocode}
\usepackage{tikz}
\usepackage{nicefrac}
\usepackage{changepage}
\usetikzlibrary{arrows, positioning, calc, graphs}

% Math
\usepackage{amsmath, amsfonts, amssymb, amsthm}
% Graphics

\usepackage{graphicx}
\usepackage{tikz}
% Code environment
\usepackage{minted}
%\usepackage{bm}
%\usepackage{siunitx}
%\usepackage{animate}
%\usepackage{hyperref}
%\usepackage{movie15}
%\usepackage{multicol}
%\usepackage{changepage}
\title{Gagnasafnsfræði Verkefni 5}
\author{Ragnar Björn Ingvarsson, rbi3}
\tikzset{->, >=stealth', shorten >=1pt, node distance=2cm,thick, main node/.style={circle,draw,minimum size=3em}}

\begin{document}
\renewcommand\thepage{}
	
	\maketitle

	\newpage
	\setcounter{page}{1}
	\renewcommand\thepage{\arabic{page}}

	\section{}
	\begin{itemize}
		\item[a)]
			\begin{itemize}
				\item[i.] $A \rightarrow B$ Gengur ekki, n-d 1 og 2 brjóta 
					regluna vegna þess að gildin í $A$ fyrir þessar n-dir 
					eru eins en eru ekki eins í $B$.
				\item[ii.] $B \rightarrow C$ Gengur, allar n-dir í $B$ gefa 
					sama í $C$ ef það er eins í $B$.
				\item[iii.] $C \rightarrow B$ Gengur ekki, n-d 1 og 3 brjóta 
					regluna vegna þess að gildin í $C$ fyrir þessar n-dir 
					eru eins en eru ekki eins í $B$.
				\item[iv.] $B \rightarrow A$ Gengur ekki, n-d 1 og 5 brjóta 
					regluna vegna þess að gildin í $B$ fyrir þessar n-dir 
					eru eins en eru ekki eins í $A$.
				\item[v.] $C \rightarrow A$ Gengur ekki, n-d 1 og 3 brjóta 
					regluna vegna þess að gildin í $C$ fyrir þessar n-dir 
					eru eins en eru ekki eins í $A$.
			\end{itemize}
		\item[b)] Sjáum að þar sem $B \rightarrow C$ er eina fallákveðan, 
			þá hljóta $A$ og $B$ að vera hluti af öllum mögulegum lyklum 
			þar sem $A$ og $B$ koma aldrei fyrir hægra megin í fallákveðum. 
			Einnig sést að $\{A,B\}^+ = \{A,B,C\}$ sem eru öll eigindin svo 
			$AB$ er eini mögulegi lykillinn.
	\end{itemize}

	\section{}
	Sjáum að eigindin sem koma ekki fram hægra megin á neinni fallákveðu 
	eru $C$ og $D$ svo bæði $C$ og $D$ hljóta að vera hluti af öllum 
	mögulegum lyklum. Einnig er $\{C,D\}^+ = \{A, B, C, D,E,F,G,H,I,J\}$ 
	sem eru öll eigindin svo $CD$ er eini mögulegi lykillinn.

	\section{}
	\begin{itemize}
		\item[a)]
			\begin{itemize}
				\item \textbf{Vensl 1} Eigindin sem koma ekki fram hægra 
					megin í neinum fallákveðum er bara $B$. $B^+ = {B}$ svo 
					við þurfum fleiri eigindi og þá sést að $\{A,B\}^+ = 
					\{B,C\}^+ = \{B,D\}^+ = \{A,B,C,D\}$ svo $AB$, $BC$, 
					$BD$ eru mögulegir lyklar.
				\item \textbf{Vensl 2} Eigindin sem koma ekki fram hægra 
					megin í neinum fallákveðum er bara $A$ og við sjáum að 
					$A^+ = \{A,B,C,D\}$ svo $A$ er eini mögulegi lykillinn.
			\end{itemize}
		\item[b)] Yfirlyklar eru safn af dálkum sem innihalda mögulega 
			lykla svo allir yfirlyklarnir sem eru ekki mögulegir lyklar eru
			\begin{itemize} 
			\item \textbf{Vensl 1} $ABC, ABD, ABCD, BCD$
				\item \textbf{Vensl 2} $AB, AC, AD, ABC, ABD, ACD, ABCD$
			\end{itemize}
			Röð eiginda skiptir ekki máli þar sem hver yfirlykill er safn, 
			þar sem röð staka skiptir ekki máli.
	\end{itemize}

	\section{}
	\begin{itemize}
		\item[$BCNF$] Byrjum á að finna alla mögulega lykla, sjáum að $B$ er 
			eina eigindið sem er ekki hægra megin í neinum fallákveðum svo 
			$B$ er hluti allra mögulega lykla. $B^+ = \{A,B,C,D\}$ svo $B$ 
			er eini mögulegi lykillinn. Sjáum að fallákveðan $C \rightarrow 
			AD$ brýtur regluna um $BCNF$ form þar sem $C$ er ekki yfirlykill.

			Skiptum þá venslinu í $R_1(A,C,D)$ og $R_2(B,C)$ og þá er 
			lykillinn í $R_1$ $C$ og lykillinn í $R_2$ $B$ og komið er á 
			$BCNF$ form.

			Einnig gildir þessi skipting fyrir $3NF$ form þar sem fyrir 
			sérhverja fallákveðu $\bar{A} \rightarrow B$ er $\bar{A}$ 
			yfirlykill.
	\end{itemize}
\end{document}

\documentclass{article}
\usepackage{graphicx} % Required for inserting images
\usepackage[top=0.9in, bottom=1in, left=1.5in, right=1.5in]{geometry}
\usepackage[utf8]{inputenc}
\usepackage[icelandic]{babel}
\usepackage[T1]{fontenc}
\usepackage[sc]{mathpazo}
\usepackage[parfill]{parskip}
\renewcommand{\baselinestretch}{1.2}
\usepackage{booktabs,tabularx}
\usepackage{multirow}
\usepackage{enumerate}
\usepackage{adjustbox}
\usepackage{multicol}
\usepackage{xcolor}
\usepackage{algpseudocode}
\usepackage{tikz}
\usepackage{nicefrac}
\usepackage{changepage}
\usetikzlibrary{arrows, positioning, calc, graphs}
\usepackage{amsmath, amsfonts, amssymb, amsthm}
\usepackage{graphicx}
\usepackage{tikz}
\usepackage{minted}
\usemintedstyle{manni}
\title{Formal Languages and Computability 9}
\author{Ragnar Björn Ingvarsson, rbi3}
\tikzset{->, >=stealth', shorten >=1pt, node distance=2cm,thick, main node/.style={circle,draw,minimum size=3em}}

\begin{document}
\renewcommand\thepage{}

	\maketitle

	\newpage
	\setcounter{page}{1}
	\renewcommand\thepage{\arabic{page}}

	\section{}
	We can just see that for every even number we can execute the function 
	\begin{equation}
		f(n) = \frac{n}{2}^2
		\label{eq:gamer}
	\end{equation}
	Which maps each even number to a perfect square. Since this function 
	is computable and provides the needed reduction, we see that $B$ is 
	mapping reducible from $A$.

	\section{}
	We can show that $A$ is decidable by giving a description for a TM that 
	decides on $A$.

	Let $M$ be the TM, which takes in a pair of natural numbers $(a,b)$. 
	We first calculate $a^2 + b^2$ and store that. Then we start with $c=1$ 
	and iterate over all natural numbers, where for each one we calculate 
	$c^2$ and compare it to our $a^2 + b^2$. If they are equal, we accept, 
	otherwise we continue running. We can then just check if $c^2$ is larger 
	than $a^2 + b^2$ and if so we reject.
	\section{}
	We show that $EQ_{CFG}$ is undecidable, we can construct a reduction 
	from $ALL_{CFG}$. We do so by first assuming a decider $M$ for 
	$EQ_{CFG}$ and then, for each grammar $G$, we construct a grammar $G_1$ that generates all 
	possible strings $\Sigma^*$. Then we use $M$ to decide if 
	$L(G) = L(G_1)$ and if it accepts, we accept, otherwise we reject. 

	From here we see that we have reduced $ALL_{CFG}$ to $EQ_{CFG}$ which 
	means that $EQ_{CFG}$ is also undecidable.

	\section{}
	We will prove this by creating a reduction from $A_{TM}$ to $F$.
	Assume that a decider $R$ exists for $F$ and we will create a decider 
	$S$ for $A_{TM}$.

	Let $S$ run on input $\langle M,w\rangle$, and then we construct an 
	encoding for a TM $\langle M_w\rangle$ which, for any input $x$, runs $M$ on $w$. 
	If $M$ accepts, $M_w$ accepts, otherwise it rejects. Then we 
	can run $R$ on $\langle M_w\rangle$ and if it accepts, $S$ rejects and 
	otherwise it accepts.

	Here we have reduced $A_{TM}$ to $F$, meaning that since $A_{TM}$ is 
	known to be undecidable, $F$ also has to be undecidable. This works on 
	the basis that if $M$ accepts $w$, $\langle M_w\rangle$ accepts all 
	possible strings, which is an infinite set, so $R$ rejects it, meaning 
	we can accept in turn. However, if $M$ doesnt accept $w$ or loops, 
	$\langle M_w\rangle$ rejects every possible string, meaning 
	$L(M_w) = \varnothing$ which is a finite set, meaning $R$ accepts it 
	so $S$ can reject.
\end{document}

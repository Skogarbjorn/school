\documentclass{article}
\usepackage{graphicx} % Required for inserting images
\usepackage[top=0.9in, bottom=1in, left=1.5in, right=1.5in]{geometry}
\usepackage[utf8]{inputenc}
\usepackage[icelandic]{babel}
\usepackage[T1]{fontenc}
\usepackage[sc]{mathpazo}
\usepackage[parfill]{parskip}
\renewcommand{\baselinestretch}{1.2}
\usepackage{booktabs,tabularx}
\usepackage{multirow}
\usepackage{enumerate}
\usepackage{adjustbox}
\usepackage{multicol}
\usepackage{xcolor}
\usepackage{algpseudocode}
\usepackage{tikz}
\usepackage{nicefrac}
\usepackage{changepage}
\usetikzlibrary{arrows, positioning, calc, graphs}
\usepackage{amsmath, amsfonts, amssymb, amsthm}
\usepackage{graphicx}
\usepackage{tikz}
\usepackage{minted}
\usemintedstyle{manni}
\title{Formal Languages and Computability 11}
\author{Ragnar Björn Ingvarsson, rbi3}
\tikzset{->, >=stealth', shorten >=1pt, node distance=2cm,thick, main node/.style={circle,draw,minimum size=3em}}

\begin{document}
\renewcommand\thepage{}

	\maketitle

	\newpage
	\setcounter{page}{1}
	\renewcommand\thepage{\arabic{page}}
	\renewcommand\thesubsection{\alph{subsection}.}

	\section{You have implemented an algorithm using a 3-tape Turing machine $X$ that runs in time $O(n^2)$. What is the time complexity when $X$ is 
	simulated with a single-tape Turing machine.}
	We observe that no matter the number of tapes, if a Turing machine 
	exists that runs on $X$ in polynomial time, all other Turing machines 
	also run on $X$ in at most polynomial time.

	We also see that every $t(n)$ time multitape Turing machine has 
	an equivalent $O(t^2(n))$ time single-tape Turing machine which 
	means that when $X$ is simulated with a single-tape Turing machine, 
	the time complexity is $O((n^2)^2)$ = $O(n^4)$.

	\newpage
	\section{Determine which of the following formulae are \\ satisfiable}
	\subsection{$(x_1 \lor x_2) \land (x_1 \lor \overline{x}_2)$} 
	This is satisfiable since we can say that $x_1 = T$, and then $x_2$ 
	is irrelevant.

	\subsection{$(x_1 \lor x_2) \land (x_1 \lor \overline{x}_2)
	\land (\overline{x}_1 \lor x_2) \land (\overline{x}_1 \lor 
	\overline{x}_2)$}
	This is not satisfiable because the first clause states that either 
	$x_1$ or $x_2$ are true. Then the last states that one of them is 
	false. Therefore we conclude that they are either true and false or 
	false and true. But the middle clauses state that both cases must 
	be true meaning that this cannot be satisfied.

	\subsection{$(\overline{x}_1\lor\overline{x}_2)\land
	(x_1\lor x_2\lor x_3)\land(\overline{x}_1\lor\overline{x}_3)$}
	This is satisfiable since we can simply state that $x_1$ is false and 
	$x_2$ is true. Then, the first and last clause are true and the middle
	also. The value of $x_3$ is then irrelevant.

	\newpage
	\section{A presidential election takes place with two parties, $A$ and 
	$B$. There are $n$ polling stations and both parties have nominated 
	volunteers to act as election monitors. Each volunteer can attend only 
one station from some subset of stations. Both parties demand to have at 
least one own election monitor in each station.}

    \subsection{Formulate the corresponding language $ELECTION$.}
	We can say that the language takes in the tuple 
	$(A, B, n)$ where $A$ is a set of subsets which contain the polling 
	stations each representative from $A$ can monitor and $B$ is the same 
	but for party $B$, and $n$ is the number of polling stations.

	The language is then defined as such

	For each $i\in\{1,2,..,n\}$ there exists a subset in $A$ and $B$
	which contains $i$, where we can express this link as the tuple 
	$(i, a_j, b_k)$, where $\forall a\in A\setminus\{a_j\}: a_j \neq a$ and\\
	$\forall b\in B\setminus\{b_k\}: b_k \neq b$.
	
	\subsection{Give a polynomial time reduction to $SAT$.}
	We can express this in two parts. First we will create the satisfiable 
	expression for ensuring that all polling stations are monitored, then 
	we will ensure that no volunteer monitors two stations or more.

	We simply express the first part as 
	\begin{equation}
		\bigwedge\limits_{s\in S} (\bigvee\limits_{a\in A} x_{sa}) \wedge 
			(\bigvee\limits_{b\in B} x_{sb})
		\label{eq:gamer}
	\end{equation}
	Where $x_{ij}$ is true if a volunteer $j$ monitors station $i$.

	The second part is quite tricky. We will consider the fact that for 
	every volunteer that can monitor a station, all pairs of volunteers 
	must contain a volunteer that does not monitor the station. Therefore 
	we can express this as
	\begin{equation}
		\bigwedge\limits_{a\in A, \alpha\in A\setminus\{a\}} 
		(\overline{x}_{a} \vee \overline{x}_{\alpha})
		\label{eq:gamer1}
	\end{equation}

	We can then repeat this for $B$ and add them both to our expression to 
	get the answer:

	\begin{equation}
		\bigwedge\limits_{s\in S}\left( (\bigvee\limits_{a\in A} x_{sa}) \wedge 
			(\bigvee\limits_{b\in B} x_{sb})
			\bigwedge\limits_{a\in A, \alpha\in A\setminus\{a\}} 
			(\overline{x}_{sa} \vee \overline{x}_{s\alpha})
			\bigwedge\limits_{b\in B, \beta\in B\setminus\{b\}} 
			(\overline{x}_{sb} \vee \overline{x}_{s\beta})\right)
		\label{eq:gamer2}
	\end{equation}

	\newpage
	\section{Let us consider a board game called the Geothermal Park. This 
	game is played on an $n\times m$ board. The game master first blocks 
some squares out, then the player is given a set of $k$ construction blocks 
with sizes $1\times h_i$ that can be rotated. The task is to build a 
continuous path using non-overlapping blocks through the area. The path 
must start at the bottom and end at the top.}

    \subsection{Formulate the corresponding language $PARK$.}
	The language takes in a tuple $(n, m, C, B)$ where $n$ is the length 
	and $m$ is the height of the game board, $C$ is the set of cells that 
	are not blocked and $B$ is the set of numbers that describe the block 
	lengths.

	To formulate the language, we need a few conditions.
	\begin{itemize}
		\item[i.] There exists a path $P = \{p_1,p_2,...,p_l\}$ where 
			each $p_i = (x_i,y_i)$ is a cell on the board such that 
			$y_1 = 1$ and $y_l = m$, and each consecutive cell in $P$ is 
			adjacent to the last, $|x_i - x_{i+1}| + |y_i - y_{i+1}| = 1$, 
			and lastly, $P\subseteq C$.
		\item[ii.] There exists a set of blocks $T$ where $T_i$ is the 
			set of cells occupied by block $b_i$. Then $T_i\subseteq P$ and 
			$T_i\cap T_j = \varnothing$ for all $i\neq j$ and lastly 
			$P = \bigcup\limits_{i}T_i$
		\item[iii.] For each block with coordinates $(x_{ij},y_{ij})\in T_i$ 
			we require that \\$(\forall j\in T_i: |x_{ij} - x_{i,j+1}| = 1) 
			\vee  (\forall j\in T_i: |y_{ij} - y_{i,j+1}| = 1)$. By 
			doing this we ensure that every block is either connected by 
			cells horizontally or vertically.
	\end{itemize}
	From these conditions we can create the language as such
	\begin{equation}
		PARK = \{(n,m,C,B)\mid \text{Conditions i., ii. and iii. are satisfied}\}
		\label{eq:gamerere}
	\end{equation}
	\subsection{Show that $PARK$ is in NP.}
	To show this we create a verifier $V$ described like such:

	On an input $(n,m,C,K,P)$ we first say that if $P\nsubseteq C$ we reject. 
	Then we create a set $S$ which contains all possible sets of 
	splits of $P$, where every cell is accounted for, into sets of lengths. These lengths represent chains of
	cells in $P$ that are adjacent either vertically or horizontally.

	If $(L\subseteq K)\in S$ we accept, otherwise reject.
	\newpage
	\subsection{Show that $PARK$ is NP-complete.}
	We have already shown that $PARK$ is in NP, so we will show that 
	$PARK$ is NP-hard by showing that it is polynomial time reducible to 
	some NP-complete problem. 

	Let's reduce from $SAT$, and I will only give high level descriptions 
	here. Let's set the conditions:

	\begin{itemize}
		\item[i.] There exists a tile in the bottom row and top row that 
			is included in the path. 
		\item[ii.] For every block length, there exist such many tiles 
			that we can express as being in a "block".
		\item[iii.] For every block in the path, every tile in each block 
			is distinct from any other tile in any other block.
		\item[iv.] Each block is either vertical or horizontal, meaning 
			each tile is either adjacent vertically or horizontally to its 
			neighboring tile.
		\item[v.] Any tile of every block must be adjacent to a tile of 
			some other block.
		\item[vi.] Every tile is in $C$.
	\end{itemize}

	We notice here that this forces all blocks to be used, but we can simply 
	run this on the powerset of $B$ to circumvent the trouble. Every one 
	of these conditions can be expressed as a $SAT$ which means the whole 
	problem can be expressed as a $SAT$, meaning it is polynomial time 
	reducible to $SAT$, a known NP-complete problem, so $PARK$ is 
	NP-hard, as well as being in NP, which shows that it is NP-complete.

\end{document}

\documentclass{article}
\usepackage{graphicx} % Required for inserting images
\usepackage[top=0.9in, bottom=1in, left=1.5in, right=1.5in]{geometry}
\usepackage[utf8]{inputenc}
\usepackage[icelandic]{babel}
\usepackage[T1]{fontenc}
\usepackage[sc]{mathpazo}
\usepackage[parfill]{parskip}
\renewcommand{\baselinestretch}{1.2}
% Tables and lists
\usepackage{booktabs,tabularx}
\usepackage{multirow}
\usepackage{enumerate}
\usepackage{adjustbox}
\usepackage{multicol}
\usepackage{xcolor}
\usepackage{algpseudocode}
\usepackage{tikz}
\usepackage{nicefrac}
\usepackage{changepage}
\usepackage{fancyvrb}
\usepackage{xlop}
\usepackage{hyperref}
\usetikzlibrary{arrows, positioning, calc, graphs}

% Math
\usepackage{amsmath, amsfonts, amssymb, amsthm}
% Graphics

\usepackage{graphicx}
\usepackage{tikz}
% Code environment
\usepackage{minted}
%\usepackage{bm}
%\usepackage{siunitx}
%\usepackage{animate}
%\usepackage{hyperref}
%\usepackage{movie15}
%\usepackage{multicol}
%\usepackage{changepage}
\title{Tölvugrafík Verkefni 1}
\author{Ragnar Björn Ingvarsson, rbi3}
\tikzset{->, >=stealth', shorten >=1pt, node distance=2cm,thick, main node/.style={circle,draw,minimum size=3em}}


\begin{document}
\renewcommand\thepage{}
	
	\maketitle

	\begin{center}\url{https://skogarbjorn.github.io/h2/test1/index.html}\end{center}
	\newpage
	\setcounter{page}{1}
	\renewcommand\thepage{\arabic{page}}

	\part{Verkefnislýsing}

	Beðið var um að gera leikinn duck hunt í webgl með nokkrum forsendum. 
	Það þarf að hafa þríhyrning sem spilara, sem við getum stjórnað með 
	því að ýta á hann og draga með músinni. Fuglar (ferhyrningar) eiga að 
	fljúga yfir skjáinn á mismunandi hraða, og spilari á að getað skotið 
	skoti (ferhyrningur) með því að ýta á takka, sem flýgur beint upp. 
	Ef skotið klessir á fugl, þá hverfur bæði fuglinn og skotið og við 
	fáum stig sem er strik efst í glugga. Ef við erum komin með 5 stig þá 
	endar leikurinn.

	\part{Verkefnið mitt}

	Ég notaði eitt \texttt{points} fylki fyrir alla punkta sem þarf að 
	geyma og hef nokkrar breytur fyrir hvar fuglarnir, skotin og stigin 
	byrja í fylkinu. Á þennan hátt er þægilegt að halda utan um hvernig 
	skipulag fylkisins er með því að hækka viðeigandi breytur þegar skot 
	eða fugl er bætt við fylkið.

	Byrjum á \texttt{game} fallinu. Ég einfaldlega teikna fyrst það sem 
	er í buffernum, svo uppfæri ég hvert skot fyrir sig, svo fuglana, og 
	að lokum reyni ég að búa til nýjan fugl með Math.random() og tékka 
	hvort það sé minna en 0.01.

	\texttt{render} fallið virkar svo að ég teikna fyrst spilarann, semsagt 
	fyrstu þrjá punktana sem þríhyrning, svo hvert skot fyrir sig, 
	sem ég nota shot\_index í, sama með fuglana og svo stigin, allt í 
	mismunandi for lykkjum til að halda hlutum aðskildum.

	Ég einnig nota spacebar til að skjóta skotum, og hef sett takmörkunina 
	að ekki er hægt að hafa fleiri en þrjú skot á skjánum í einu.

	Að uppfæra staðsetningu fuglanna og skotanna er nokkuð einfalt, ég 
	bara fer í gegn um alla punkta sem tilheyra fuglunum og bæti smá gildi 
	við x gildið. Eins með skotin. Síðan tékka ég hvort þau séu farin af 
	skjánum og ef svo er fyrir eitthvert skot eða fugl, þá tek ég 
	punktana úr points með splice fallinu og uppfæri allar index breytur. 
	Svo nota ég break til að vera viss um að ekki ruglast þar sem 
	ég breyti points á meðan ég er í for lykkju sem fer í gegn um points.

	Fyrir collision þá nota ég AABB algorithmann, og tékka fyrir hvert 
	skot, fyrir alla fugla, ef skotið er komið á efri helming skjásins 
	(þar sem fuglarnir eru). Ef eitthvert skot klessist á við einhvern 
	fugl, þá eyði ég báðum og breaka þar sem skotið sem ég var að tékka 
	á er nú farið.

	Ég lenti í nokkrum örðugleikum við að forrita þetta, en aðallega 
	var vesen að láta fuglana fljúga á mismunandi hraða. Ég leysti það 
	með því að hafa annað fylki sem heldur utan um hraða allra fuglanna. 
	Vesenið kemur þó af því hvernig ég á að tengja hvern fugl þegar ég 
	uppfæri staðsetningu þeirra við hvaða hraða hann á. Til að koma til 
	móts við þessu notaði ég sérhæfðar for lykkjur sem byrja á 0 og hækka 
	upp í fjölda fugla. Síðan, inni í lykkjunni, þá nota ég alltaf 
	$birds\_index + i*4$ til að færa fuglinn, og er svo með i sem er index 
	fuglsins í hraða-fylkinu.

	Loks, þá tékka ég í hverri lykkju af \texttt{game}, hvort stigin eru færri en 
	fimm, og ef þau eru fimm eða hærri þá kalla ég á render einu sinni enn 
	til að hafa lokaskjáinn réttan, og svo endurtek ekki aftur \texttt{game}.

	\part{Viðbætur}

	Ég semsagt útfærði allar viðbæturnar sem lýst var að væri hægt að 
	bæta við leikinn í verkefnislýsingunni. Forritið getur höndlað 
	nær ótakmarkað marga fugla á skjánum í einu, og þeir fljúga allir í 
	mismunandi hæð og á mismunandi hraða. Einnig er hægt að hafa upp að 
	þremur skotum á skjánum í einu og þau virka öll eins og þau eiga að 
	gera. Og loks birtast strik (mjóir ferhyrningar) efst í hægra horni 
	skjásins þegar skot hittir fugl, og ef fimm strikum er náð þá endar 
	leikurinn.


\end{document}

\documentclass{article}
\usepackage{graphicx} % Required for inserting images
\usepackage[top=0.9in, bottom=1in, left=1.5in, right=1.5in]{geometry}
\usepackage[utf8]{inputenc}
\usepackage[icelandic]{babel}
\usepackage[T1]{fontenc}
\usepackage[sc]{mathpazo}
\usepackage[parfill]{parskip}
\renewcommand{\baselinestretch}{1.2}
\usepackage{booktabs,tabularx}
\usepackage{multirow}
\usepackage{enumerate}
\usepackage{adjustbox}
\usepackage{multicol}
\usepackage{xcolor}
\usepackage{algpseudocode}
\usepackage{tikz}
\usepackage{nicefrac}
\usepackage{changepage}
\usetikzlibrary{arrows, positioning, calc, graphs}
\usepackage{amsmath, amsfonts, amssymb, amsthm}
\usepackage{graphicx}
\usepackage{tikz}
\usepackage{minted}
\usemintedstyle{manni}
\title{Tölvugrafík Verkefni 3}
\author{Ragnar Björn Ingvarsson, rbi3 \\
Elías Ver Bjarnason, evb17}
\tikzset{->, >=stealth', shorten >=1pt, node distance=2cm,thick, main node/.style={circle,draw,minimum size=3em}}

\begin{document}
\renewcommand\thepage{}

	\maketitle
	\begin{center}\url{https://skogarbjorn.github.io/v3redo/index.html}\end{center}

	\newpage
	\setcounter{page}{1}
	\renewcommand\thepage{\arabic{page}}

	\section{Verkefnislýsing}
Í þessu forritunarverkefni á að útfæra einfalda
þrívíddarútgáfu af hinum sígilda tölvuleik Frogger. Í leiknum
á að koma litlum frosk yfir umferðargötu án þess að bílarnir
keyri á hann og síðan á staurum yfir á án þess að hann detti
út í ánna

\section{Okkar lausn}
Við notuðum semsagt THREE.js og byrjuðum á $16x13$ grind sem froskurinn getur hreyft sig um á, 
og byrjum leikinn á því að gefa bílum, staurum og skjaldbökum staðsetningu af handahófi 
og geymum þá í fylkjum sem eru skipt í raðir svo hægt sé að komast framhjá 
óþarfa tékkum fyrir collision seinna meir. Síðan höfum við það þannig að 
hver röð er með sér hraða geymdir í \texttt{laneSpeed} fylkinu og uppfærum 
þannig staðsetningu allra hluta í hverju render \texttt{tick()} kalli, sem 
gerist á $10$ms fresti. Við látum svo hlutina birtast hinum megin þegar þeir 
fara út af leikborðinu. 

Við ákváðum að hafa myndavélina þannig að hún helst fyrir aftan spilarann í miðjunni 
og færist áfram með honum en horfir á hann ef hann færist til hliðar. 
Svo er að sjálfsögðu hægt að færa sig inn í sjónarhorn frosksins og þá 
látum við hann hverfa svo við teiknum hann ekki og notum \texttt{PointerLockControls} frá THREE.js til að færa sjónarhornið til. 
Myndavélin horfir alltaf á kubbinn á staðnum sem hann er teiknaður en 
raun-hnitin hans hækka og lækka alltaf bara um heilan reit í einu. Við 
notum svo \texttt{smooth} fallið til að færa teiknaða spilarann um helming 
í átt að raun-hnitunum hverjar $10$ms svo það verði þægilegt að horfa á 
hann hreyfast.

Síðan erum við með mjög lágar líkur í hverju \texttt{tick} kalli að einhver 
skjaldbaka fari í kaf. Þá getur hún ekki farið í kaf aftur á meðan hún er 
að kafa og fer línulega niður og svo rís aftur eftir nokkrar sekúndur. 

Sama virknin er í raun með flugurnar, við höfum bara smá séns að fluga 
birtist í hverju \texttt{tick} kalli og þá gefum við henni staðsetningu 
af handahófi en bara fyrir efri helming leikborðsins. Síðan erum við með 
fylki sem heldur utan um allar flugur og við einfaldlega athugum alltaf 
hvort spilari hafi náð flugu og ef svo er tökum við hana úr \texttt{scene} 
og hækkum stigin. 

Svo klárar spilari leikinn með því að koma fjórum froskum alla leið yfir 
og fylla í öll fjögur hólfin. Þá er leik lokið og stig prentast út í 
console.

Við bjuggum síðan til nokkur einföld módel til að setja inn og notum bara 
\texttt{GLTFLoader} frá THREE.js til að hlaða öllu inn og það er í raun 
ekki mikið flóknara en það.


\end{document}

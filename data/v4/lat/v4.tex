\documentclass{article}
\usepackage{graphicx} % Required for inserting images
\usepackage[top=0.9in, bottom=1in, left=1.5in, right=1.5in]{geometry}
\usepackage[utf8]{inputenc}
\usepackage[icelandic]{babel}
\usepackage[T1]{fontenc}
\usepackage[sc]{mathpazo}
\usepackage[parfill]{parskip}
\renewcommand{\baselinestretch}{1.2}
% Tables and lists
\usepackage{booktabs,tabularx}
\usepackage{multirow}
\usepackage{enumerate}
\usepackage{adjustbox}
\usepackage{multicol}
\usepackage{xcolor}
\usepackage{algpseudocode}
\usepackage{tikz}
\usepackage{nicefrac}
\usepackage{changepage}
\usetikzlibrary{arrows, positioning, calc, graphs}

% Math
\usepackage{amsmath, amsfonts, amssymb, amsthm}
% Graphics

\usepackage{graphicx}
\usepackage{tikz}
% Code environment
\usepackage{minted}
%\usepackage{bm}
%\usepackage{siunitx}
%\usepackage{animate}
%\usepackage{hyperref}
%\usepackage{movie15}
%\usepackage{multicol}
%\usepackage{changepage}
\title{Gagnasafnsfræði Verkefni 4}
\author{Ragnar Björn Ingvarsson, rbi3}
\tikzset{->, >=stealth', shorten >=1pt, node distance=2cm,thick, main node/.style={circle,draw,minimum size=3em}}

\def\ojoin{\setbox0=\hbox{$\bowtie$}%
  \rule[-.02ex]{.25em}{.5pt}\llap{\rule[\ht0]{.25em}{.5pt}}}
\def\leftouterjoin{\mathbin{\ojoin\mkern-5.8mu\bowtie}}
\def\rightouterjoin{\mathbin{\bowtie\mkern-5.8mu\ojoin}}
\def\fullouterjoin{\mathbin{\ojoin\mkern-5.8mu\bowtie\mkern-5.8mu\ojoin}}

\begin{document}
\renewcommand\thepage{}
	
	\maketitle

	\newpage
	\setcounter{page}{1}
	\renewcommand\thepage{\arabic{page}}

	Látum $\omega$ tákna NULL gildi.

	Einnig má minnast á að í SQLITE þá er röð eiginda x,y,z,w,t en samkvæmt 
	formúlunni frá wikipedia:
	\begin{equation}
	U = \Pi_{r_1,...,r_n,c_1,...,c_m,s_1,...,s_k}(P)
	\end{equation}
	fæst að röð þeirra eigi að vera x,z,y,w,t. Ég vel 
	hér að nota röðun SQLITE.


	\section{$R\bowtie S$}
	\begin{center}
	\begin{tabular}{|c|c|c|c|c|}
		\hline
		X & Y & Z & W & T \\
		\hline
		15 & c & 8 & 10 & 6 \\
		15 & c & 8 & 10 & 5 \\
		\hline
	\end{tabular}
	\end{center}
	
	\section{$R\leftouterjoin S$}
	\begin{center}
	\begin{tabular}{|c|c|c|c|c|}
		\hline
		X & Y & Z & W & T \\
		\hline
		10 & b & 5 & $\omega$ & $\omega$ \\
		15 & c & 8 & 10 & 6 \\
		15 & c & 8 & 10 & 5 \\
		25 & b & 6 & $\omega$ & $\omega$ \\
		\hline
	\end{tabular}
	\end{center}
	\section{$R\rightouterjoin S$}
	\begin{center}
	\begin{tabular}{|c|c|c|c|c|}
		\hline
		X & Y & Z & W & T \\
		\hline
		15 & c & 8 & 10 & 6 \\
		15 & c & 8 & 10 & 5 \\
		$\omega$ & d & $\omega$ & 25 & 3 \\
		\hline
	\end{tabular}
	\end{center}
	\section{$R\fullouterjoin S$}
	\begin{center}
	\begin{tabular}{|c|c|c|c|c|}
		\hline
		X & Y & Z & W & T \\
		\hline
		10 & b & 5 & $\omega$ & $\omega$ \\
		15 & c & 8 & 10 & 6 \\
		15 & c & 8 & 10 & 5 \\
		25 & b & 6 & $\omega$ & $\omega$ \\
		$\omega$ & d & $ \omega$ & 25 & 3 \\
		\hline
	\end{tabular}
	\end{center}
	\section{$R\bowtie_{R.Y=S.Y\land Z=W}S$}
	\begin{center}
	\begin{tabular}{|c|c|c|c|c|}
		\hline
		X & Z & Y & W & T \\
		\hline
	\end{tabular}
	\end{center}

\end{document}

\documentclass{article}
\usepackage{graphicx} % Required for inserting images
\usepackage[top=0.9in, bottom=1in, left=1.4in, right=1.4in]{geometry}
\usepackage[utf8]{inputenc}
\usepackage[icelandic]{babel}
\usepackage[T1]{fontenc}
\usepackage[sc]{mathpazo}
\usepackage[parfill]{parskip}
\renewcommand{\baselinestretch}{1.2}
\usepackage{booktabs,tabularx}
\usepackage{multirow}
\usepackage{enumerate}
\usepackage{adjustbox}
\usepackage{multicol}
\usepackage{xcolor}
\usepackage{algpseudocode}
\usepackage{tikz}
\usepackage{nicefrac}
\usepackage{changepage}
\usetikzlibrary{arrows, positioning, calc, graphs}
\usepackage{amsmath, amsfonts, amssymb, amsthm}
\usepackage{graphicx}
\usepackage{tikz}
\usepackage{minted}
\usemintedstyle{manni}
\title{gamer}
\author{Ragnar Björn Ingvarsson, rbi3}
\tikzset{->, >=stealth', shorten >=1pt, node distance=2cm,thick, main node/.style={circle,draw,minimum size=3em}}

\begin{document}
\renewcommand\thepage{}

	\maketitle

	\newpage
	\setcounter{page}{1}
	\renewcommand\thepage{\arabic{page}}

	\section{}
	\section{}
	\section{}
	\section{Endurtakið reikningana í 3a, d og e fyrir n = 77 og e = 7}
	\begin{itemize}
		\item[a)] Upplýsingar í opinberum lykli eru $(n,e)$ en í 
			einkalykli eru $(n,d)$ (eða $(p,q)$, bæði virkar)

			Fáum skv. lýsingu að $n=77$ og $e=7$, svo við reiknum út $p$ og 
			$q$:
			\[
				n = 77 = 7\cdot 11
			\]
			Svo við segjum að $p=11$ og $q=7$, og reiknum $d$:
			\begin{equation}
				d \equiv e^{-1}\text{ mod } (p-1)(q-1)
				\label{eq:gamer1}
			\end{equation}
			Og við leysum þá andhverfuna:
			\[de \equiv 1\text{ mod } 10\cdot6\]
			\[\iff 7d\equiv 1\text{ mod }60\]
			Sjáum að fyrir $d=43$ fæst $7\cdot43 = 301$ sem gefur $1$ 
			mátað við $60$, svo við margföldum í gegn með $43$ og fáum
			\[43\cdot 7\cdot d \equiv 43\cdot1\text{ mod } 60\]
			\[\iff d \equiv 43\text{ mod }60\]
			Svo við segjum að $d = 43$ og opinberi lykillinn er þá 
			$(77,7)$ og einkalykillinn er $(77,43)$.
		\item[d)] Við dulkóðum skilaboð $m$ skv.
			\begin{equation}
				c \equiv m^e\text{ mod }n
				\label{eq:gamer2}
			\end{equation}
			Svo við fáum að dulkóðuðu skilaboðin $c$ eru
			\[c \equiv 3^7\text{ mod }77\]
			\[\iff c\equiv 2187\text{ mod }77\]
			\[\iff c\equiv 31\text{ mod }77\]
			Svo $c = 31$.
		\item[e)] Til að afkóða skilaboð þarf að nota formúluna
			\begin{equation}
				m \equiv c^d\text{ mod }n
				\label{eq:gamer3}
			\end{equation}
			Svo við stingum inn gildunum og fáum
			\[m\equiv 31^{43}\text{ mod }77\]
			\[\iff m\equiv 31\cdot(37^{21})\text{ mod }77\]
			Og höldum áfram svona og loks fæst
			\[m\equiv 3\text{ mod }77\]
			Svo $m=3$.
	\end{itemize}
\end{document}

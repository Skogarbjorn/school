\documentclass{article}
\usepackage{graphicx} % Required for inserting images
\usepackage[top=0.9in, bottom=1in, left=1.5in, right=1.5in]{geometry}
\usepackage[utf8]{inputenc}
\usepackage[icelandic]{babel}
\usepackage[T1]{fontenc}
\usepackage[sc]{mathpazo}
\usepackage[parfill]{parskip}
% Tables and lists
\usepackage{booktabs,tabularx}
\usepackage{multirow}
\usepackage{enumerate}
\usepackage{adjustbox}
\usepackage{multicol}
\usepackage{xcolor}
\usepackage{algpseudocode}
\usepackage{tikz}
\usepackage{hyperref}
\usetikzlibrary{arrows, positioning, calc}

% Math
\usepackage{amsmath, amsfonts, amssymb, amsthm}
% Graphics

\usepackage{graphicx}
\usepackage{tikz}
% Code environment
\usepackage{minted}
\hypersetup{colorlinks=true}
\title{Virðistilboð (E1)}
\author{Ragnar Björn Ingvarsson, rbi3}


\begin{document}
	
	\maketitle
	
	\section*{Um Hopp}

	Hopp er fyrirtæki sem býður upp á rafmagnshlaupahjól (rafskútu) hvar sem er  
	á helstu stöðum Íslands. Þessi þjónusta gerir fólki kleift að öðlast 
	farartæki, sem er mitt á milli þess að ganga og að fara í strætó, 
	í skyndi hvar sem það megi vera. Hopp varð til árið 2019 og hefur 
	afar fljótt orðið þekkt og notað um allt land.

	\section*{Virðistilboð}

	Virðistilboð er í raun það virði sem fyrirtæki býður upp á fyrir 
	markaðshópinn sinn. Það getur verið þjónusta eða vara fyrir venjulegt fólk, 
	þjónusta tengd öðrum fyrirtækjum, og alls kyns fleiri gerðir, en í 
	grunninn þarf það að skapa virði. Virði getur komið alls staðar frá og
	er afar margþætt. Virði er til dæmis skapað í gegn um ónýttan markað, 
	markaðssetningu og góða þjónustu.

	Hopp býr til virði sitt með því að bjóða upp á leið til að ferðast 
	stuttar leiðir sem er bæði hentug og umhverfislega góð. Rafskútur 
	þeirra myndu flokkast sem bæði vara og þjónusta þar sem þjónusta þeirra er að hafa rafskútu 
	tiltæka nærri hvar sem er og varan er þá rafskútan sjálf. 

	\vspace{3mm}

	Hopp skapar virði sitt á nokkra vegu:

	\vspace{3mm}

	\begin{itemize}
		\item[\textbf{Tímasetning:}] Vegna þess að Hopp var fyrsta fyrirtækið til að slá á þennan 
			markað á Íslandi hefur það búið skjótt til virði og hefur 
			haldið því með því að bæði hafa meiri tíma til að fullkomna 
			þjónustu sína en aðrir á markaðnum og með því að hafa 
			puttann á púlsinum og þróast stöðugt og betur en keppendur 
			hafa.
		\item[\textbf{Kolefnisfótspor:}] Hopp segist einnig vera umhverfisvænt og grænt fyrirtæki 
			sem er aðlagandi fyrir fólk sem er meðvitað um kolefnisfótsporið
			sitt. Þetta gefur þeim virði umfram bíla eða vespur og gerir 
			Hopp einnig kleift að skera sig inn í markað hjóla og 
			hlaupahjóla. Einnig mun aukin notkun á Hopp hjólum hvetja ríkið 
			til þess að byggja fleiri vegi og innviði fyrir hjól og rafskútur sem mun 
			minnka notkun bíla og þar með kolefnisneyslu Íslands í heild 
			sinni.
		\item[\textbf{Markaðssetning:}] Markaðssetningin þeirra hjálpar gríðarlega að skapa virði,
			 og því er náð vegna auðþekkjanlega sægræna litar þeirra og
			orkumikilla auglýsinga sem tengja fólk, aðallega ungmenni 
			að skemmta sér, við Hopp. Þetta gefur þeim vinsældir meðal 
			yngri hópa og myndar jákvæða tengingu milli ungs fólks og Hopp,
			 sem aðrir keppendur virðast ekki gera jafn vel.
		\newpage
		\item[\textbf{Hugbúnaður:}] Hopp hefur þróað hagkvæman og hentugan hugbúnað sem gerir 
			fólki kleift að leiga skútu og byrja ferð með nær engum núningi 
			við tæknina. Miðað við samkeppnina er þetta ómissandi tækni og 
			myndar ofsalegt virði fyrir Hopp.
	\end{itemize}

	\vspace{3mm}

	Í lokin stendur Hopp þá út meðal keppenda sem fyrirmynd markaðsins, 
	sem kom um með því að vera brautryðjendur rafskúta á Íslandi, með 
	hugbúnað byggðan á nýjustu tækninni,
	sköpuðu heilbrigt og 
	grænt vistkerfi sem miðar á að byggja bjartari framtíð, og klárri markaðssetningu miðuð á ungt fólk sem mun bera þessa framtíð á öxlum sér. 

	\vspace{1cm}

	Allar upplýsingar fengnar frá heimasíðu Hopp: \url{https://hopp.bike/is}

\end{document}


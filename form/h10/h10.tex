\documentclass{article}
\usepackage{graphicx} % Required for inserting images
\usepackage[top=0.9in, bottom=1in, left=1.5in, right=1.5in]{geometry}
\usepackage[utf8]{inputenc}
\usepackage[icelandic]{babel}
\usepackage[T1]{fontenc}
\usepackage[sc]{mathpazo}
\usepackage[parfill]{parskip}
\renewcommand{\baselinestretch}{1.2}
\usepackage{booktabs,tabularx}
\usepackage{multirow}
\usepackage{enumerate}
\usepackage{adjustbox}
\usepackage{multicol}
\usepackage{xcolor}
\usepackage{algpseudocode}
\usepackage{tikz}
\usepackage{nicefrac}
\usepackage{changepage}
\usetikzlibrary{arrows, positioning, calc, graphs}
\usepackage{amsmath, amsfonts, amssymb, amsthm}
\usepackage{graphicx}
\usepackage{tikz}
\usepackage{minted}
\usemintedstyle{manni}
\title{Formal Languages and Computability 10}
\author{Ragnar Björn Ingvarsson, rbi3}
\tikzset{->, >=stealth', shorten >=1pt, node distance=2cm,thick, main node/.style={circle,draw,minimum size=3em}}

\begin{document}
\renewcommand\thepage{}

	\maketitle

	\newpage
	\setcounter{page}{1}
	\renewcommand\thepage{\arabic{page}}

	\section{}
	We see that the big $O$ of $f(n)$ is $O(n^2)$.
	\begin{itemize}
		\item[a)] This grows slower than $n^2$ so this is true.
		\item[b)] $2^n$ grows faster than $n^2$ so this is false.
		\item[c)] $n^4$ grows faster than $n^2$ so this is false.
		\item[d)] We see that this depends on whether $log_2n^n$ grows 
			faster than $n$, and we can rewrite $log_2n^n$ as 
			$n + log_2\frac{n}{2}^n$ which we can see grows faster than $n$ 
			so this is true.
		\item[e)] Since $2^n$ grows faster than $n^2$ is this true.
	\end{itemize}
	So a), d) and e) are true.

	\section{}
	\begin{itemize}
		\item[a)] We see that each revolution takes $n + n + 2 = 2n + 2$ 
			steps, and the number of revolutions is $\frac{n}{4}$ since 
			if the length of the string is $n$, then the number of dibits 
			is $\frac{n}{2}$ and then the number of pairs of dibits is 
			$\frac{n}{4}$. Then we just subtract the final rewind step 
			since that isnt executed on the final step and we get 
			$\frac{n}{4}(2n+2) - n = \frac{1}{2}(n^2 - n)$.
		\item[b)] No, since we get
			\[
				\lim_{n\to\infty} \frac{n^2}{\nicefrac{1}{2}(n^2 - n)} = 
				\lim_{n\to\infty} \frac{n^2}{n^2(\nicefrac{1}{2} - \nicefrac{1}{2n})} 
			\]
			\[
				= \lim_{n\to\infty} \frac{1}{\nicefrac{1}{2} - \nicefrac{1}{2n}}
				= \frac{1}{\nicefrac{1}{2}} = 2 \neq 0
			\]
			So $n^2$ grows at a faster pace than $\frac{1}{2}(n^2 - n)$.
		\item[c)] A faster version of this algorithm would be to have
			it blank out the left side pairs as soon as we begin the search 
			for a pair, so we travel back and forth the tape with an average 
			of only $n$ steps, making the exact running time 
			$\frac{n}{4}(n + 2) - \frac{n}{2} = \nicefrac{n^2}{4}$
	\end{itemize}

	\newpage
	\section{}
	\begin{itemize}
		\item[a)] Tabulating the times for $n = 1,...,7$ we get

			\begin{center}
				\begin{tabular}[c]{|l|l|l|l|l|l|l|l|}
					\hline
					\textbf{Length} & \textbf{1} & \textbf{2} & \textbf{3} & \textbf{4} & \textbf{5} & \textbf{6} & \textbf{7} \\
					\hline
					\hline
					\textbf{Steps} & $2$ & $3$ & $6$ & $13$ & $28$ & $59$ & $122$ \\
					\hline
				\end{tabular}
			\end{center}
		
		\item[b)] We simply experiment enough until we get the formula 
			\begin{equation}
				r(n+1) = 2r(n) + n - 2
				\label{eq:gamer}
			\end{equation}
		\item[c)] To find the exact closed form expression we will use 
			the formula for a linear first order difference equation of the 
			form $x(n+1) = a(n)x(n) + c(n)$,

			\begin{equation}
				y(n) = \left( \prod_{k=1}^{n-1} a(k) \right)y_0 
				+ \sum_{k=1}^{n-1}\left( \prod_{j=k+1}^{n-1}a(j)\right)c(k)
				\label{eq:gamer1}
			\end{equation}

			Where we get

			\[
				r(n) = \left( \prod_{k=1}^{n-1} 2 \right)\cdot 2
				+ \sum_{k=1}^{n-1}\left( \prod_{j=k+1}^{n-1}2\right)\cdot(k-2)
			\]
			\[
				= 2^n + \sum_{k=1}^{n-1} k2^{n-k-1} - 2^{n-k}
			= 2^n + 2^{n-1}\sum_{k=1}^{n-1}\frac{k}{2^k} - 2^n\sum_{k=1}^{n-1}\frac{1}{2^k}
			\]
			\[
				= 2^n - n + 1
			\]
			So the closed form is $r(n) = 2^n - n + 1$.
	\end{itemize}

	\section{}
	So the algorithm goes like this, we first mark one node, then, we 
	iterate through each node in $G$ to check if it is connected to 
	any marked node and if so, we mark it. We then continue until no 
	new nodes are marked. At last we iterate through each node and check 
	if any remain unmarked, if so we reject otherwise we accept.

	Looking at the worst (not even possible) case of this, we get that 
	we have to iterate through every node (n times) and for each node 
	we assume that they are connected to every other node, meaning we 
	check $n-1$ paths for each node. We then do all this a maximum of 
	$n-1$ times since at least one node must be marked every time, and 
	then one more time to confirm everything which leaves it at $n$ times.

	We then get that the theoretical maximum amount of steps for this is 
	$n * (n-1) * n = n^3 - n^2$, which is clearly in $P$.

\end{document}

\documentclass{article}
\usepackage{graphicx} % Required for inserting images
\usepackage[top=0.9in, bottom=1in, left=1.5in, right=1.5in]{geometry}
\usepackage[utf8]{inputenc}
\usepackage[icelandic]{babel}
\usepackage[T1]{fontenc}
\usepackage[sc]{mathpazo}
\usepackage[parfill]{parskip}
\renewcommand{\baselinestretch}{1.2}
\usepackage{booktabs,tabularx}
\usepackage{multirow}
\usepackage{enumerate}
\usepackage{adjustbox}
\usepackage{hyperref}
\usepackage{multicol}
\usepackage{xcolor}
\usepackage{algpseudocode}
\usepackage{tikz}
\usepackage{nicefrac}
\usepackage{changepage}
\usetikzlibrary{arrows, positioning, calc, graphs}
\usepackage{amsmath, amsfonts, amssymb, amsthm}
\usepackage{graphicx}
\usepackage{tikz}
\usepackage{minted}
\usemintedstyle{manni}
\title{Tölvugrafík Heimadæmi 5}
\author{Ragnar Björn Ingvarsson, rbi3}
\tikzset{->, >=stealth', shorten >=1pt, node distance=2cm,thick, main node/.style={circle,draw,minimum size=3em}}

\begin{document}
\renewcommand\thepage{}

	\maketitle

	\newpage
	\setcounter{page}{1}
	\renewcommand\thepage{\arabic{page}}

	\section{}
	\url{https://skogarbjorn.github.io/h5/1/hus.html}
	\section{}
	\url{https://skogarbjorn.github.io/h5/2/run.html}

	\section{}
    \begin{itemize}
    	\item[a)] Við sjáum að þar sem dreifendurskin er bjartast þar sem 
			ljós skín beint á yfirborð, svo bjartasta dreifendurskinið á 
			þessari mynd er í punkti $C$. Svo er bjartasta depilendurskinið 
			örugglega rúmlega í botni dalsins, þar sem endurskinsvigurinn 
			er næstum jafn sjónvigrinum.
    	\item[b)] Fyrir alla punktana er sama umhverfisendurskin þar sem 
			styrkleiki ljóssins helst eins og endurkastsstuðullinn $k$ er 
			sá sami fyrir allt yfirborðið.

			Fyrir $A$ er dreifendurskinið mjög sterkt þar sem þvervigurinn 
			er næstum jafn ljósvigrinum, en depilendurskinið er því sem næst 
			ekkert þar sem hornið milli sjónvigurs og endurskinsvigurs er 
			rúmlega $90$ gráður.

			Fyrir $B$ er dreifendurskinið ekkert 
			þar sem hornið milli þvervigursins og ljósvigursins er stærra en 
			$90$ gráður, og depilendurskinið verður einnig ekkert þar sem 
			ef ljósvigur bendir á bak við yfirborð bendir endurskinsvigurinn 
			til baka á ljósið, svo hornið milli hans og sjónvigurs verður 
			$90$ gráður sem núllar út gildið. 

			Fyrir $C$ er rosalega sterkt 
			dreifendurskin eins og útskýrt var í a), en depilendurskinið 
			frekar dauft þar sem hornið milli sjónvigurs og endurskinsvigurs 
			er svolítið stórt en þó ekki $90$ gráður.
    \end{itemize}

	\section{}
	\begin{itemize}
		\item[a)] Þetta er satt, eins og sést í formúlu fyrir dreifendurskin 
			er aldrei hugsað um staðsetningu áhorfanda, 
			\begin{equation}
				I_d = k_d(l\cdot n)L_d
				\label{eq:diffuse}
			\end{equation}
		\item[b)] Þetta er ósatt, umhverfisendurskin er einungis notað til 
			að nálga óbeina lýsingu vegna endurkasts, ekki fyrir nálgun 
			annarra ljósgjafa.
		\item[c)] Þetta er ósatt þar sem dofnunargildi fer bara eftir lengd 
			ljósgjafa frá yfirborði, ekki tengt áhorfanda.
		\item[d)] Þetta er einnig ósatt, litur umhverfisendurskins ætti að 
			fara eftir lit yfirborðs en litur depilendurskins ætti að vera 
			sami og litur ljósgjafa.
	\end{itemize}
\end{document}
